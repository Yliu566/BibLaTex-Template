%%%%%%%%%%%%%%%%%%%%%%%%%%%%%%%%%%%%%%%%%%%%%%%%%%%%%%%%%%%%%%%%%%%%
%% Place any additional packages needed here.  Only include packages
%% which are essential, to avoid problems later.
%%%%%%%%%%%%%%%%%%%%%%%%%%%%%%%%%%%%%%%%%%%%%%%%%%%%%%%%%%%%%%%%%%%%%
\usepackage[version=4]{mhchem} % Formula subscripts using \ce{}
\usepackage[separate-uncertainty=true]{siunitx}
\sisetup{detect-all}
\usepackage{enumerate}
\usepackage{xcolor}
\usepackage{color, colortbl}
\usepackage{subfig}
\usepackage{xr} % For cross-referencing figures
%\usepackage[bookmarks=true,linkbordercolor={1 1 1},citebordercolor={1 1 1},urlbordercolor={1 1 1}]{hyperref}
\hypersetup{pdfauthor = {R.C. Chiechi et al},
pdftitle = {Manuscript for submission},
pdfsubject = {Manuscript for Submission},
%pdfkeywords = {Keyword1, Keyword2, ...},
pdfcreator = {LaTeX with hyperref package},
colorlinks, breaklinks, linkcolor=black, urlcolor=black, anchorcolor=black, citecolor=black
}

%
% BibLaTeX Setup
%
\usepackage[utf8]{inputenc}
\usepackage[style=chem-acs]{biblatex}
\addbibresource{references.bib}
%
%

%%%%%%%%%%%%%%%%%%%%%%%%%%%%%%%%%%%%%%%%%%%%%%%%%%%%%%%%%%%%%%%%%%%%%
%% Place any additional macros here.  Please use \newcommand* where
%% possible, and avoid layout changing macros (which are not used
%% when typesetting).
%%%%%%%%%%%%%%%%%%%%%%%%%%%%%%%%%%%%%%%%%%%%%%%%%%%%%%%%%%%%%%%%%%%%%

\newcommand{\red}[1]{\textcolor{red}{#1}}
\newcommand{\blue}[1]{\textcolor{blue}{#1}}
\newcommand{\green}[1]{\textcolor{green}{#1}}
\newcommand*{\ie}{\textit{i.e.}, }
\newcommand*{\eg}{\textit{e.g.}, }

\newcommand*{\ts}[1]{$\mathrm{#1}^{\mathrm{TS}}$}
\newcommand*{\mica}[1]{$\mathrm{#1}^{\mathrm{Mica}}$}
\newcommand*{\cp}[1]{$\mathrm{#1}^{\mathrm{AFM}}$}

\newcommand*{\fermi}{$E_\mathrm{f}$}
\newcommand*{\egap}{$E_\mathrm{g}$}
\newcommand*{\Junits}{\si{Acm^{-2}}}
\newcommand*{\logJ}{$\log|J|$}
\newcommand*{\logI}{$\log|I|$}
\newcommand*{\vtrans}{$V_{\mathrm{trans}}$}
\newcommand*{\vtransp}[1]{$V_{\mathrm{trans}}^{#1}$}
\newcommand*{\degC}[1]{\SI{#1}{\celsius}}
\newcommand*{\et}{\textit{et al}.}

% Declare molar units with siunitx
\DeclareSIUnit\Molar{\textsc{M}}

% Examples of macros for defining molecule abbreviations
\newcommand*{\Tn}[1]{\textbf{C4T#1}}
\newcommand*{\TnC}{C4T\textit{n}}
\newcommand*{\Thn}{T\textit{n}}
\newcommand*{\OPEn}{OPE\textit{n}}
\newcommand*{\BDT}[1]{\textbf{BDT-#1}}

%macros for this paper
\newcommand*{\DTT}{\textbf{DTT}}     %DTT
\newcommand*{\dDTT}{\textbf{\textit{d}-DTT}}  %dimerised DTT
\newcommand*{\bpDTT}{\textbf{(bp)\textit{d}-DTT}}  %bidentate physisorbed DTT
\newcommand*{\mpDTT}{\textbf{(mp)\textit{d}-DTT}}  %monodentate physisorbed DTT
\newcommand*{\bcDTT}{\textbf{(bc)-DTT}}  %bidentate chemisorbed DTT
\newcommand*{\C}[1]{\textbf{C#1}}           %use as \C{0}, \C{1}, etc.
\newcommand*{\Cn}{\textbf{C\textit{n}}}     %lipoic acid, just use as \Cn
\newcommand*{\EtSH}{\textbf{EtSH}}     %Ethanethiol
\newcommand*{\AuS}{\textbf{Au--S}}     %Au-S bond
\newcommand*{\AuSS}{\textbf{Au$\dotsm$(S--S)}}     %Au--S-S bond
\newcommand*{\AuSSAu}{\textbf{Au$\dotsm$(S--S)$\dotsm$Au}}     %Au--S-S bond
\newcommand*{\SSbond}{\textbf{S--S}}     %S-S bond
\newcommand*{\AuSH}{\textbf{Au$\dotsm$SH--R}}     %Au-SH physisorbed bond
\newcommand*{\AuAu}{\textbf{Au--Au}}     %Au-Au bond
\newcommand*{\F}{figure}
\newcommand*{\m}[1]{\textbf{#1m}}     %S-S bond
